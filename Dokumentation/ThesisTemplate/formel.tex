%---------------------------------------------
%
%	formel.tex
%
%---------------------------------------------
	%
\chapter*{Formelzeichen und Abk"urzungen}
	%
	%
%
%-------------------------------------------------------
\begin{tabbing}
	%
ABCDEF\=Erl"auterung\kill\\
	%
$\korresp$	\> Korrespondenz zwischen einem Zeitsignal und seinem Spektrum\\
$\lfloor x.y \rfloor$ \> ganzzahliger Anteil von $x.y$ \lt $x $\\
$\lceil x.y \rceil$ \> kleinster ganzzahliger Wert gr"o"ser gleich $x.y$ \lt $\left\{ \begin{array}{cl}
					x 	& , y=0\\
					x+1	& , y\neq 0
					\end{array}\right.$\\
$<\ul{a}, \ul{b}>$	\> Skalarprodukt der Vektoren $\ul{a}$ und $\ul{b}$\\
$[t_k,t_{k+1})$	\> Intervall in den Grenzen von einschlie"slich $t_k$ bis ausschlie"slich $t_{k+1}$\\
$a_n, a[n$]	\>Approximations- oder Tiefpasskoeffizient \\
$x[n]$		\>zeitdiskretes Signal, Folge von Symbolen	\\
$X(z)$		\>zeitdiskretes Signal im $\cal{Z}$-Bereich	\\
%$zs$		\>Anzahl der Zerlegungsstufen bei der dyadischen Wavelet-Transformation \\
${\mathds{Z}}$	\> Menge der ganzen Zahlen
	%
\end{tabbing}
	%
\vfil
	%
%Abkuerzungen
%-------------------------------------------------------
\begin{tabbing}
	%
ABCDEF\=Erl"auterung\kill\\
	%
AKF	\> Autokorrelationsfunktion\\
CODEC	\> enCOder/DECoder \\
CQF	\> Konjugiert-Quadratur-Filterbank (engl.: {\em conjugate quadrature filterbank}) \\
MPEG	\> Motion Picture Experts Group	\\
MSE	\> mittlerer quadratischer Fehler (engl.: {\em mean square error}) \\
MUX	\> Multiplexer \\
	%
\end{tabbing}
	%
