%---------------------------------------------
%
%   hauptteil.tex
%
%---------------------------------------------
	%
%----------------------------------------------
\chapter{Hauptteil -- Beispiel}
%----------------------------------------------
	%
{\em Hier wird alles beschrieben, was w�hrend der Bearbeitung des Themas durchgef�hrt wurde. Wichtig dabei ist das Darstellen der Entscheidungsfindung. Fast alle Problemstellungen k�nnen auf verschiedene Weise gel�st werden. Es ist zu begr�nden,  warum eine L�sung der anderen vorgezogen wurde.  Eine Ingenieursleistung ist immer durch selbst�ndiges Denken, Entscheiden und Handeln gekennzeichnet. Das beinhaltet auch, dass man Literatur zu Rate zieht und entsprechend referenziert \cite{UWQ08}.  Leistung hei�t unter anderem auch Arbeit pro Zeit.}
	%
%----------------------------------------------
\section{Spezifizierung der Problemstellung}
%----------------------------------------------
	%

Der Hauptteil der Arbeit sollte ca.\ die H�lfte bis zu zwei Dritteln des Geschriebenen ausmachen. Er kann sich durchaus in mehrere Kapitel untergliedern.


%----------------------------------------------
\section{Beschreibung der Baugruppen und Methoden}
%----------------------------------------------
	%
Beim Formulieren eines Themas ist nicht immer im Voraus klar, ob die Untersuchungen und Entwicklungen zum Erfolg f�hren. Auch R�ckschl�ge sind zu dokumentieren, damit dieser Weg f�r sp�tere Arbeiten ausgeschlossen werden kann.

%----------------------------------------------
\section{Implementierung}
%----------------------------------------------
	%
Manchmal kann es hilfreich sein, programmiertechnische Details zu beschreiben, wenn zum Beispiel ein besonderer Algorithmus zum Implementieren einer Methode erforderlich war. Dann, und nur dann, muss Quelltext mit in die Dokumentation aufgenommen werden.
Das Quellcode-Beispiel in \ref{lst_test123} zeigt, wie man einzelne Elemente in einem eindimensionalen Array zweidimensional addressieren kann.
	%
	\begin{lstlisting}[caption={Einfache Adressieren von Bildpunkten}\label{lst_test123},captionpos=t]
	int	val;	/* Integer-Wert	*/
	unsigned int x, y;
	unsigned int width, height;	/* Breite und H�he des Bildes	*/
	:
	:
	for ( y = 0; y < height; y++) /* Schleife �ber alle Zeilen	*/
	{
		for ( x = 0; x < width; x++)  /* Schleife �ber alle Spalten	*/
		{
			pos = x + y * width;
			val = bild[pos]; /* Wert an der Position [y,x]	*/
			:
		}
	}
	:
	\end{lstlisting}
	
Das Listing in in \ref{lst_adressierung} zeigt wie es besser (schneller) geht.
	%
\lstinputlisting
    [caption={Verbessertes Adressieren von Arrays}
       \label{lst_adressierung}, captionpos=t,language=C]
 {Quellcode/adressierung.c}	% Enbinden �ber externe Datei	
	
	%
	
Es ist zu sehen, dass die zweite Variante das Berechnen von {\tt pos} deutlich vereinfacht, weil Multiplikationen nicht mehr erforderlich sind und auch die Addition nur einmal pro Zeile erfolgen muss ({\tt py += width}).
	%
%----------------------------------------------
\section{Ergebnisse}
%----------------------------------------------
	%
Der Hauptteil ist sinnvoll zu gliedern. Den Abschluss bildet oft eine Pr�sentation von Ergebnissen (Diagramme, Messreihen, Kurven, etc.), welche ausgewertet und mit Schlussfolgerungen erg�nzt werden.
	%
%----------------------------------------------
%----------------------------------------------
