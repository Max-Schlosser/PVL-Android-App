\documentclass[11pt]{scrartcl}
 
\usepackage{ucs}
\usepackage[utf8x]{inputenc}
\usepackage[T1]{fontenc}
\usepackage[ngerman]{babel}
\usepackage{dsfont}
\usepackage[table,xcdraw]{xcolor}
%\usepackage{longtable}

\title{Dokumentation PVL ICT - QuanPic}
\author{Björn S. Beuran, Florian Hollas, Tom Kober, Gianluca Ragusa,\\ Max Schlosser, Stephan Wagener}
\date{\today}

\begin{document}
\maketitle
\tableofcontents
\thispagestyle{plain}
\clearpage
 
 \section{Projektziel – tbc}
 Das Ziel des Projekts ist es, bis zur 22. KW, eine App für die Android Version 4.xx zu entwerfen,
 welche „ein Bild von der Smartphone-Kamera aufnimmt und live auf dem Display anzeigt. Vor der 
 Anzeige sollen die Bilddaten modifiziert werden können. Der RGB-Farbraum soll quantisiert werden. 
 Basierend auf einer Literaturrecherche sind mindestens zwei verschiedene Quantisierungsverfahren 
 zu implementieren und zu vergleichen. Das modifizierte Bild muss gespeichert werden können.“
 \\Des Weiteren, sollen Programmcode, Texte und Testbilder abgegeben werden. Hinzu kommt die 
 entworfene .apk-Datei.
 
 \section{Untersuchung}
 \subsection{Quantisierung - tdc}
 Die Quantisierung von Bildern lässt sich in drei Schritte gliedern.
 \\Hierzu gehören die Erzeugung digitaler Bilder, die Repräsentation von Bilddaten und die Speicherung 
 dieser.
 \subsection{Funktionsweise Android Studio - Stephan}
 Hier soll ein Text von Stephan stehen.
 \subsection{OpenCV - Tom}
 Hier soll ein Text von Tom stehen.
 \subsection{Android App Programmierung - Alle}
 Hier soll ein Text von allen stehen.
 \section{Vorgehensweise}
 
 \subsection{Zuständigkeiten}
 




\begin{longtable}
\begin{tabular}{lllll}
\hline
\multicolumn{5}{l}{\cellcolor[HTML]{9B9B9B}Koordination}                                                                                           \\ \hline
Schlosser     & Max             & s133229    & \multicolumn{2}{l}{\begin{tabular}[c]{@{}l@{}}- Aufgabe a\\ - Aufgabe b\\ - Aufgabe c\end{tabular}} \\ \hline
\multicolumn{5}{l}{\cellcolor[HTML]{9B9B9B}Programmierung}                                                                                         \\ \hline
Hollas        & Florian         & s133215    & \multicolumn{2}{l}{\begin{tabular}[c]{@{}l@{}}- Aufgabe a\\ - Aufgabe b\\ - Aufgabe c\end{tabular}} \\ \hline
Kober         & Tom             & s133219    & \multicolumn{2}{l}{\begin{tabular}[c]{@{}l@{}}- Aufgabe a\\ - Aufgabe b\\ - Aufgabe c\end{tabular}} \\ \hline
Schlosser     & Max             & s133229    & \multicolumn{2}{l}{\begin{tabular}[c]{@{}l@{}}- Aufgabe a\\ - Aufgabe b\\ - Aufgabe c\end{tabular}} \\ \hline
Wagener       & Stephan         & s133235    & \multicolumn{2}{l}{\begin{tabular}[c]{@{}l@{}}- Aufgabe a\\ - Aufgabe b\\ - Aufgabe c\end{tabular}} \\ \hline
\multicolumn{5}{l}{\cellcolor[HTML]{9B9B9B}Dokumentation}                                                                                          \\ \hline
Beuran        & Björn Simion    & s133202    & \multicolumn{2}{l}{\begin{tabular}[c]{@{}l@{}}- Aufgabe a\\ - Aufgabe b\\ - Aufgabe c\end{tabular}} \\ \hline
Hollas        & Florian         & s133215    & \multicolumn{2}{l}{\begin{tabular}[c]{@{}l@{}}- Aufgabe a\\ - Aufgabe b\\ - Aufgabe c\end{tabular}} \\ \hline
Ragusa        & Gianluca        & s133131    & \multicolumn{2}{l}{\begin{tabular}[c]{@{}l@{}}- Aufgabe a\\ - Aufgabe b\\ - Aufgabe c\end{tabular}} \\ \hline
\end{tabular}
\end{longtable}




 

 \subsection{Versionierung - Alle Programmierer}
 Hier soll ein Text von allen Programmierern stehen.
 \section{Quellen}
 Von allen zu Pflegen!
 At the moment:
 \\http://www.gm.fh-koeln.de/~konen/WPF-BV/BV03.PDF
 \section{Notizen}
 Hier könnt ihr eure aktuellen Anmerkungen bezüglich der Doku reinschreiben
 
\end{document}
